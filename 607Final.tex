\documentclass{article}

% if you need to pass options to natbib, use, e.g.:
%     \PassOptionsToPackage{numbers, compress}{natbib}
% before loading neurips_2020

% ready for submission
%\usepackage{neurips_2020}

% to compile a preprint version, e.g., for submission to arXiv, add add the
% [preprint] option:
\usepackage[preprint]{neurips_2020}

% to compile a camera-ready version, add the [final] option, e.g.:
%     \usepackage[final]{neurips_2020}

% to avoid loading the natbib package, add option nonatbib:
%\usepackage[nonatbib]{neurips_2020}

\usepackage[utf8]{inputenc} % allow utf-8 input
\usepackage[T1]{fontenc}    % use 8-bit T1 fonts
\usepackage{hyperref}       % hyperlinks
\usepackage{url}            % simple URL typesetting
\usepackage{booktabs}       % professional-quality tables
\usepackage{amsfonts}       % blackboard math symbols
\usepackage{nicefrac}       % compact symbols for 1/2, etc.
\usepackage{microtype}      % microtypography
\usepackage{hyperref}
\usepackage[usenames,dvipsnames]{xcolor}
\usepackage{amsmath, mathtools}
\usepackage{listings}
\lstdefinelanguage{Julia}%
  {morekeywords={abstract,break,case,catch,const,continue,do,else,elseif,%
      end,export,false,for,function,immutable,import,importall,if,in,%
      macro,module,otherwise,quote,return,switch,true,try,type,typealias,%
      using,while},%
   sensitive=true,%
   alsoother={$},%
   morecomment=[l]\#,%
   morecomment=[n]{\#=}{=\#},%
   morestring=[s]{"}{"},%
   morestring=[m]{'}{'},%
}[keywords,comments,strings]%
%
\lstset{%
    language         = Julia,
    basicstyle       = \ttfamily,
    keywordstyle     = \bfseries\color{blue},
    stringstyle      = \color{magenta},
    commentstyle     = \color{ForestGreen},
    showstringspaces = false,
}


\title{CIS 607 Final Project}

% The \author macro works with any number of authors. There are two commands
% used to separate the names and addresses of multiple authors: \And and \AND.
%
% Using \And between authors leaves it to LaTeX to determine where to break the
% lines. Using \AND forces a line break at that point. So, if LaTeX puts 3 of 4
% authors names on the first line, and the last on the second line, try using
% \AND instead of \And before the third author name.

\author{%
    Steven Walton\\
  \texttt{swalton2@cs.uoregon.edu} \\
  % examples of more authors
  % \And
  % Coauthor \\
  % Affiliation \\
  % Address \\
  % \texttt{email} \\
  % \AND
  % Coauthor \\
  % Affiliation \\
  % Address \\
  % \texttt{email} \\
  % \And
  % Coauthor \\
  % Affiliation \\
  % Address \\
  % \texttt{email} \\
  % \And
  % Coauthor \\
  % Affiliation \\
  % Address \\
  % \texttt{email} \\
}

\begin{document}

\maketitle

%\begin{abstract}
%\end{abstract}

Herein contains my final project for CIS607. This paper contains all the code
and notes related to the assignment (with section 3 being the ``bonus'' question
as stated by the latest email). The code can also be found at
\url{https://github.com/stevenwalton/CIS607-GPUProgramming} for added convenience.

\section{Create \textit{my\_solvers.jl}}

Create \textit{my\_solvers.jl} and include a function \textit{conj\_grad()} that
performs the conjugate gradient (CG) algo- rithm (serial, CPU).
\textit{conj\_grad()} should take in a positive definite matrix A, an initial
guess, a right-hand-side vector b, a tolerance  $\epsilon$ (set to $10^{-6}$)
and a maximum number of iterations $iter_{max}$ (normally set to a multiple of the
length of b), and return an approximate solution $\tilde{x}$ to $Ax = b$, such
that the relative error $err_R \leq \epsilon$ where
\[
    err_R = \frac{||A\tilde x - b||}{||\tilde x||}
\]

where $||\cdot||$ is the 2-norm, i.e. $||x|| = \sqrt{x^T x}$. (Hint: one way to
generate a positive definite $N \times N$ matrix is to define $A = I + B^T B$,
as suggested above. A common choice for initial condition is the vector of all
zeros).

\subsection{My Code: \textit{my\_solvers.jl}}
Note that the actual code contains a few comments and a block commented out
section for testing this code

\begin{lstlisting}
using LinearAlgebra
using CUDA

function conj_grad(A, x, b, epsilon, iter_max)
    r = b - A*x
    err = norm(r)
    if err < epsilon * norm(x)
        return x
    end
    p = r
    for k = 1:iter_max
        r_old = r
        alpha = (r' * r) / (p' * A * p)
        x .= x + alpha * p
        r = r - alpha * A * p
        err = norm(r) 
        if err < epsilon * norm(x)
            return x
        end
        beta = (r' * r) / (r_old' * r_old)
        p = r + beta * p
    end
    return x
end

function norm_cuda!(err, x)
    for i = 1:length(x)
        err[1] = err[1] + (x[i] * x[i])
    end
    err = CUDA.sqrt(err[1])
    return nothing
end

function conj_grad_cuda!(A, x, b, epsilon, iter_max)
    r = b - A * x
    err = CuArray([0.])
    @cuda norm_cuda!(err, r)
    err2 = CuArray([0.])
    @cuda norm_cuda!(err2, x)
    if any(err .< epsilon .* err2)
        return 
    end
    p = copy(r)
    for k = 1:iter_max
        r_old = copy(r)
        alpha = (r' * r) / (p' * A * p)
        x .= x + alpha * p
        @cuda norm_cuda!(err, r)
        @cuda norm_cuda!(err2, x)
        if any(err .< epsilon .* err2)
            return 
        end
        beta = (r' * r) / (r_old' * r_old)
        p = r + beta * p
    end
    return 
end
\end{lstlisting}

\section{Poisson Equation}
The 1-dimensional Poisson equation with Dirichlet boundary conditions is given
by
\begin{align*}
    -u_{xx} &= F(x), & 0 \leq x \leq 1\\
    u(0) &= 0\\
    u(1) &= 0
\end{align*}

where the source $F(x) = \pi^2 \sin(\pi x)$

\subsection{a) \textit{poisson1D.jl}}

Create a new Julia script and call it \textit{poisson1D.jl} and include
\textit{my\_solvers.jl}. To solve the 1D Poisson equation numerically,
discretize in space with N total nodes using the standard (second-order
accurate) centered difference approximation to $u_{xx}$, and arrive at a system of
equations $Au = b$, where $A$ is positive definite.

\subsubsection{My Code: \textit{poisson1D.jl}}
Note here that we time the whole block to get better averaging, though each step
size will take a different amount of time. But this will better compare with the
GPU code
\begin{lstlisting}

include("my_solvers.jl")

using LinearAlgebra
using SparseArrays
using Printf
using CUDA

function get_error(solution, input, step)
    u_total = [0; input; 0]
    return sqrt((u_total - solution)' * (u_total - solution)) * sqrt(step)
end

function poisson(h, eps)
    N = Integer(1/h + 1) # Total number of nodes
    m = N - 2 # total interior nodes
    x0 = zeros(m)
    x = 0:h:1
    exact = sin.(pi*x)
    A = zeros(m, m)
    for i = 1:m
        A[i, i] = 2/h^2
    end

    for i = 1:m-1
        A[i, i+1] = -1/h^2
        A[i+1, i] = -1/h^2
    end

    # make vector b
    b = Array{Float32}(undef, m)
    for i = 1:m
        b[i] =  pi^2 * sin(pi * x[i]) 
    end

    u_int = conj_grad(A, x0, b, eps, N^2)
    error = get_error(exact, u_int, h)

    # Half
    hd2 = h/2
    N_half = Integer(1 / hd2 + 1)
    m_half = N_half - 2
    x0_half = zeros(m_half)
    x_half = 0:hd2:1
    exact_half = sin.(pi * x_half)
    A_half = zeros(m_half, m_half)
    for i = 1:m_half
        A_half[i, i] = 2/hd2^2
    end
    for i = 1:m_half-1
        A_half[i, i+1] = -1/hd2^2
        A_half[i+1, i] = -1/hd2^2
    end

    # make vector b
    b_half = Array{Float32}(undef, m_half)
    for i = 1:m_half
        b_half[i] =  pi^2 * sin(pi * x_half[i]) 
    end

    u_int_half = conj_grad(A_half, x0_half, b_half, eps, N_half^2)
    error_half = get_error(exact_half, u_int_half, hd2)

    @printf "%f\t %f\t %f\t %f\n" h error (error/error_half) log2(error)
end

eps = 1e-9
h_list = [0.1, 0.05, 0.025, 0.0125]

@printf "dx\t\t error_dx\t ratio\t\t rate\n"
@time begin
    for i = 1:length(h_list)
        h = h_list[i]
        poisson(h, eps)
    end
end
\end{lstlisting}

\subsection{b) Calling \textit{conj\_grad.jl}}

 Call \textit{conj\_grad()} to solve the system in part (a). Note that the exact
 solution is given by $u(x) = \sin(\pi x)$.  Confirm that your numerical
 solution is converging in space at the correct rate (rate $\approx 2$) by
 computing a convergence table.  Where the error is the discrete L$_2$-norm of
 difference $u(x_j) - u_j$ (i.e. the exact solution $u(x)$ evaluated on the spatial
 grid and the numerical solution $u_j$ ). I will provide more details in class

\subsubsection{Filling in the table}
The total execution time took $1.443953$ seconds ($3.99$ M allocations:
$265.980$ MiB

\begin{table}[h]
  \begin{center}
    \begin{tabular}{cccc}
      \toprule
      \multicolumn{1}{c}{} &  \multicolumn{2}{c}{2nd Order}  \\
      \midrule
      $\Delta x$&  \hspace{3mm} error$_{\Delta x}$ &\hspace{3mm} ratio = error$_{\Delta x}$/error$_{\Delta x/2}$ \hspace{3mm} &\hspace{3mm} rate = $\log_{2}$(ratio)   \\
      $0.1$    & 0.055232 & 2.232350 & -4.178364 \\
      $0.05 $  & 0.024741 & 2.064825 & -5.336927 \\
      $0.025$  & 0.011982 & 2.016699 & -6.382947 \\
      $0.0125$ & 0.005942 & 2.004207 & -7.394943 \\
      \bottomrule
    \end{tabular}
  \end{center}
%  \caption{Error and convergence rates using the method of manufactured
%  solutions.\label{tab:error}}
\end{table}

Included in graph~\ref{fig:cpu} is the convergence graphs for the CPU code. Note
that we have inverted the x-axis for easier interpretation.
\begin{figure}
    \includegraphics[width=\textwidth]{"CPU.png"}
    \caption{Graphs for CPU code}
    \label{fig:cpu}
\end{figure}

\section{GPU Functionality (bonus!)}

Add GPU functionality to the conjugate gradient method in part 1. Time the GPU
code and compare to the serial version for several problems sizes (e.g. $N =
10^2,10^3,\cdots$) Profile it using nvprof or Nsight. Comment on performance
results.

\subsection{My GPU Code}
Note that I do not think my code is completely correct. This is because my code
does not seem to be converging. As far as I can tell the functions are the same.
I also check (see commented code) that the matrices are the same and that things
were being allocated correctly. I am less sure how to debug the poisson code as
I am unable to print out statements. I also found it frustrating that I was
unable to allocate memory within a function, which is why this code is formatted
differently than the CPU code. I'm not sure if there is a better way to do this
but interested in knowing if there is. 

Code to call the conjugate gradient (this is in \textit{poisson1D.jl})
\begin{lstlisting}
include("my_solvers.jl")

using LinearAlgebra
using SparseArrays
#using Plots
using Printf
using CUDA

function fillA!(A, h)
    m = size(A)[1]
    for i = 1:m
        A[i, i] = 2 / h^2
    end
    for i = 1:m-1
        A[i, i+1] = -1 / h^2
        A[i+1, i] = -1 / h^2
    end
    return nothing
end

function get_error_cuda!(error, solution, input, step)
    u_total = CuArray([0; input; 0])
    error = CUDA.sqrt((u_total - solution)' * (u_total - solution)) * sqrt(step) 
    return nothing
end


eps = 1e-9
h_list = [0.1, 0.05, 0.025, 0.0125]

@time begin
    for i = 1:length(h_list)
        h = h_list[i]
        N = Integer(1 / h + 1)
        m = N-2
        x0 = CUDA.zeros(m)
        x_host = 0:h:1
        x = x_host |> cu
        exact = sin.(pi * x_host)
        A = CUDA.zeros(m, m)
        @cuda fillA!(A, h)
        b_host = zeros(m)
        for j = 1:m
            b_host[j] = pi^2 * sin(pi * x_host[j])
        end
        b = b_host |> cu
        synchronize()
        #A_host = zeros(m, m)
        #for i = 1:m
        #    A_host[i, i] = 2/h^2
        #end
        #for i = 1:m-1
        #    A_host[i, i+1] = -1/h^2
        #    A_host[i+1, i] = -1/h^2
        #end
        dummy = CUDA.zeros(m)
        #@cuda threads=128 blocks=32 knl_gemv!(dummy, A, x)
        #dummy2 = zeros(m)
        #gemv!(dummy2, A_host, x_host)
        #@show all((Array(dummy) - dummy2) .< 0.000000001)
        conj_grad_cuda!(A, x0, b, CuArray([eps]), N^2, dummy)
        error = get_error(exact, Array(x0), h)

        # half calculation
        h_half = h / 2
        N = Integer(1 / h_half + 1)
        m = N-2
        x0_half = CUDA.zeros(m)
        x_host = 0:h_half:1
        x = x_host |> cu
        exact = sin.(pi * (0:h_half:1))
        A = CUDA.zeros(m, m)
        @cuda fillA!(A, h_half)
        b_host = zeros(m)
        for j = 1:m
            b_host[j] = pi^2 * sin(pi * x_host[j])
        end
        b = b_host |> cu
        synchronize()
        dummy_half = CUDA.zeros(m)
        conj_grad_cuda!(A, x0_half, b, CuArray([eps]), N^2, dummy_half)
        error_half = get_error(exact, Array(x0_half), h_half)
        @printf "%f\t %f\t %f\t %f\n" h error (error/error_half) log2(error)
    end
end
\end{lstlisting}

The conjugate gradient GPU code (this is also in \textit{my\_solvers.jl})
\begin{lstlisting}
using LinearAlgebra
using CUDA

function norm_cuda!(err, x)
    for i = 1:length(x)
        err[1] = err[1] + (x[i] * x[i])
    end
    err = CUDA.sqrt(err[1])
    return nothing
end

function knl_gemv!(y, A, x)

    N = length(y)

    bid = blockIdx().x  # get the thread's block ID
    tid = threadIdx().x # get my thread ID
    dim = blockDim().x  # how many threads in each block

    i = dim * (bid - 1) + tid #unique global thread ID
        if i <= N
            for k = 1:N
                y[i] += A[i, k]*x[k]
            end
        end
    return nothing
end

function conj_grad_cuda!(A, x, b, epsilon, iter_max, dummy)
    threads_pb=32
    num_blocks = 5#cld(N, threads)
    @cuda threads=threads_pb blocks=num_blocks knl_gemv!(dummy, A, x)
    #r = b - A * x
    r = b - dummy
    err = CuArray([0.])
    @cuda norm_cuda!(err, r)
    err2 = CuArray([0.])
    @cuda norm_cuda!(err2, x)
    # Need .< Because this is an "array"
    if any(err .< epsilon .* err2)
        return 
    end
    p = copy(r)
    for k = 1:iter_max
        r_old = copy(r)
        # Note that this appears to give the same results as when using
        # knl_gemv!()
        #alpha = (r' * r) / (p' * A * p)
        @cuda threads=threads_pb blocks=num_blocks knl_gemv!(dummy, A, p)
        alpha = (r' * r) / (p' * dummy)
        x .= x + alpha * p
        @cuda norm_cuda!(err, r)
        @cuda norm_cuda!(err2, x)
        # Need .< Because this is an "array"
        if any(err .< epsilon .* err2)
            return 
        end
        beta = (r' * r) / (r_old' * r_old)
        p = r + beta * p
    end
    return 
end
\end{lstlisting}

This code took $75.118018$ seconds to run ($65.55$ M allocations: $3.899$ GiB) I
think the time difference is because the arrays were not sparse (evidence given
by the 4GB memory allocation). I am unsure if CUDA can do this and the only way
I know how to solve this is by writing a specific kernel for tridiagonal
matrices and then converting A to a different format. 

\begin{table}[h]
  \begin{center}
    \begin{tabular}{cccc}
      \toprule
      \multicolumn{1}{c}{} &  \multicolumn{2}{c}{2nd Order}  \\
      \midrule
      $\Delta x$&  \hspace{3mm} error$_{\Delta x}$ &\hspace{3mm} ratio = error$_{\Delta x}$/error$_{\Delta x/2}$ \hspace{3mm} &\hspace{3mm} rate = $\log_{2}$(ratio)   \\
      $0.1$    & 0.390671 & 0.793954 & -1.355974 \\
      $0.05 $  & 0.492058 & 0.843534 & -1.023101 \\
      $0.025$  & 0.583328 & 0.910197 & -0.777620 \\
      $0.0125$ & 0.640881 & 0.952675 & -0.641871 \\
      \bottomrule
    \end{tabular}
  \end{center}
%  \caption{Error and convergence rates using the method of manufactured
%  solutions.\label{tab:error}}
\end{table}

The graph~\ref{fig:gpu} shows that the code may be diverging rather than converging. This is
unfortunately not what we want.

\begin{figure}
\includegraphics[width=\textwidth]{"GPU.png"}
    \caption{Graphs for GPU code}
    \label{fig:gpu}
\end{figure}

\end{document}
